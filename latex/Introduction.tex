\begin{document}
$n$ individuals
$K$ communities
Each individual belongs to exactly one community
Denote by $A$ the $(n,K)$ membership matrix, i.e $A_{i,j} = 1$ if the $i$-th individual belongs to the $j$-th community (and $0$ otherwise), for $1\leq i \leq n$ and $1\leq j \leq K$.
Denote by $X$ the $(n,n)$ connectivity matrix.
We use the SBM model :
SBM assumes that for each $1 \leq i,j \leq n$, $X_{i,j}$ follows a Bernoulli law whose parameter only depends on $g(i)$ and $g(j)$, the respective groups of i and j. Furthermore, it assumes that the coordinates of the matrix $X$ are all independent.
Letting $C$ be the $(K,K)$ matrix such that $C_{i,j}$ is the parameter of connectivity between groups $i$ and $j$, one can write $E[X] = ACA^T - diag(ACA^t)$.
In what follows we write $X = ACA^t + \mathcal{E} - D$,
where $\mathcal{E} = X-E[X]$ is a zero-mean matrix and $D = diag(ACA^t)$.
The objective is to recover the membership matrix $A$, up to a permutation, given one realization of $X$, i.e given one instance of connections between the n individuals.
Note that the membership matrix can be represented equivalently by the "normalized" membership $(n,n)$ matrix $B*$ defined as follows :
$B_{i,j} = \frac{1}{|G_k|}$ if $i$and $j$ both belong to group $k$
and $B_{i,j} = 0$ otherwise. 
Following the notations of [2], we now write $X = ZA^t + E$, where $Z = AC$ and $E = \mathcal{E} - D$. In doing so we can see the SBM model as a special instance of the G-latent models defined in [2]. 
This paper shows that the main guarantees and results of [2] can be successfully adapted to the SBM model.
Denoting $\Delta(C) = \min_{j<k}(C_{kk}+C_{jj}-2C_{jk})$, namely we show that under some conditions on $\Delta(C)$, one can recover the exact matrix $B*$ by solving a convex optimization problem :
Let $\mathcal{C} = \begin{Bmatrix}
  B \succeq 0
\left\ {}
\right\ \\ \Sigma_a B_{ab}=1, \forall b
 & \\ B_{ab}\geq 0,\forall a,b
 & \\ \mbox{tr}(B) = K \right\ 
 
\end{Bmatrix} 
\subset \mathbb{R}^{p\times p}$.
Let $\widehat{\Sigma}=X^tX$ and $\widehat{\Gamma} = \frac{1}{n}E^tE$.
PECOK algorithm :
1/ Estimate $B*$ by $\widehat{B} = \displaystyle{\mbox{argmax}_{B\in \mathcal{C}}<\widehat{\Sigma}-\widehat{\Gamma},B>}$
2/ Estimate $G*$ by applying a clustering algorithm to the columns of $\widehat{B}$.

In this paper, we develop suficient conditions on the SBM model, via the quantity $\Delta(C)$, so that the PECOK algorithm above recovers $B*$, and hence $G*$, exactly with high probability.

Our investigation follows the outline of [2], as its main arguments can be adapted to our case. Lemma 1 p.6 and its proof p.16 remain valid and so is Lemma 3 p.16.
So we only need to prove that $<\widehat{\Sigma}-\widehat{\Gamma},B^*-B> >0$ for all $B \in \mathcal{C} such that \mbox{supp}(B)\nsubseteq \mbox{supp}(B^*) $, with high probability.
Following the decomposition $(46)$ we write similarly $W = W_1 + W_2 -n\widehat{\Gamma}$.
[1] Lei, Rinaldo. Consistency of spectral clustering in stochastic block models.
[2] PECOK : a convex optimization approach to variable clustering.

\end{document}